\chapter{Einführung in das Thema Industrie 4.0}

Der Begriff "`Industrie 4.0\index{Industrie 4.0}"' ist seit seiner Popularisierung durch die Bundeskanzlerin auf der Hannovermesse 2013 vor allem in den Bereichen Produktion und Fertigung in aller Munde. Da er jedoch je nach Branche unterschiedliche Bedeutungen haben kann, soll als Einführung zunächst erläutert werden, was Industrie 4.0 im Kontext der Automatisierungstechnik bedeutet.
%% Nachdem im Verlauf der vorhergehenden industriellen Revolutionen Menschliche Arbeitskraft zunächst durch Maschinen ersetzt wurde(erste industrielle Revolution), diese dann im Zuge der Massenproduktion zusammengeschaltet wurden(zweite industrielle Revolution) und dann Mitte diesen Jahrhunderts durch den Einsatz von elektrischen und informationstechnischen 
Der Begriff beschreibt die vierte industrielle Revolution und die einhergehende Verflechtung von informationstechnisch erhobenen Daten in den Produktionsablauf. Ein interessanter Aspekt ist hier der Bereich der prädiktiven Wartung, bei dem Anhand der Auswertung empirischer Daten mögliche Anlagenausfälle frühzeitig erkannt und behoben werden können.
Für den weiteren Verlauf dieser Arbeit ist vor allem das sogenannte \ac{CPPS}\index{\acl{CPPS}} von Bedeutung. Dieses besteht aus der Verbindung von einzelnen, dezentralen Objekten wie Produktionsanlagen oder Logistikkomponenten, welche mit eingebetteten Systemen ausgestattet und zudem kommunikationsfähig gemacht werden. Durch eingebaute Sensoren und Aktoren kann die Umwelt erfasst und beeinflusst werden. Mittels der Kommunikationskomponenten können Daten aus der Produktion über ein Netzwerk oder das Internet ausgetauscht, beziehungsweise von entsprechenden Diensten ausgewertet, verarbeitet oder gespeichert werden
\cite{Bauerhansl2014}.
Sind mehrere \aclp{CPPS} an einem Produktionsprozess beteiligt, unabhängig von ihrem Standort, so spricht man auch von einer Smart Factory\index{Smart Factory}.
Abschließend ist noch zu erwähnen, dass der Begriff "`Industrie 4.0"' vor allem im deutschsprachigen Raum verwendet wird. Im internationalen Kontext werden viele der zentralen Punkte von Industrie 4.0 durch das Konzept des \ac{IoT}\index{\acl{IoT}} abgedeckt.
\clearpage




