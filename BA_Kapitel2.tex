\chapter{Definition der Anforderungen}

\section{Allgemeine Aufgabenstellung}

	Diese Arbeit beschäftigt sich mit dem Entwurf und der Realisierung eines \ac{CPPS} im Modellmaßstab. Die resultierende Anlage soll unter anderem dazu dienen verschiedene Aspekte von Industrie 4.0 vorzuführen und zu veranschaulichen. Kernpunkte, die dargestellt werden sollen, sind vor allem die Dezentralisierung und Skalierbarkeit der Anlage. Den Rahmen für die Bearbeitung dieser Aufgabe bildet die Fachberatung für Automatisierungstechnik der Siemens AG in München. Um die erwähnten Konzepte demonstrieren zu können, soll die Anlage gemäß ihrer realen Vorbilder bestehen aus Bearbeitungsstationen, an welchen der Bearbeitungsprozess simuliert werden kann, und Werkstückträgern, welche Werkstücke durch die Modellanlage zu den Maschinenplätzen transportieren können. Hauptaufgabe des Modells ist es zu zeigen, wie ein Werkstück seine Fertigung selbst organisiert. Zu Beginn des Fertigungsprozesses wird dem Werkstück ein Fertigungsplan übergeben, anhand welchem es sich durch die Anlage bewegt. Im vorliegenden Fall stellen die Werkstückträger gleichzeitig das Werkstück dar, das die Produktionsanlage durchfährt. Ebenso werden die Bearbeitungsstationen nur symbolisch dargestellt durch entsprechende Markierungen in der Anlagentopologie. Anhand verschiedener Use-cases und Szenarien soll es nach Fertigstellung der Anlage möglich sein, verschiedene Aspekte von Industrie 4.0 anschaulich darzustellen, zum Beispiel zu Vorführzwecken bei Kunden.

\section{Aufteilung der Themenbereiche}

	Wie soeben beschrieben konzentriert sich die Aufgabe vor allem auf die Entwicklung und Implementierung der Werkstückträgerfahrzeuge. Da dies jedoch noch immer ein recht allgemein gefasstes Themengebiet ist und der zur Realisierung notwendige Aufwand nur schwer abzuschätzen ist, wurde entschieden, die Hauptaufgabe in zwei Teilaufgaben zu unterteilen. Diese beiden Themenblöcke sind "`Autonomes Fahren in industriellen Transportsystemen"' und "`Dynamische Wegfindung in einer flexiblen Produktionsanlage"'\cite{I40Modell}.Die Dokumentation der Implementierung der dynamischen Wegfindung ist Hauptbestandteil der vorliegenden Arbeit. Den Teil des autonomen Fahrens mittels eines \ac{FTS} wird von Herrn Andreas Meier bearbeitet und wird deshalb nur kurz umrissen.

\subsection{Fahrerloses Transportsystem}
	Hauptaufgabe der Werkstückträger, die jeweils durch ein \ac{FTF} realisiert werden sollen, ist es, einen Weg anhand einer vorgegebenen Route ab zu fahren und somit die Werkstücke zwischen den einzelnen Bearbeitungsschritten zur nächsten Station oder Maschine zu transportieren. Herausforderungen hierbei sind die Spurführung und die Navigation innerhalb der Anlagentopologie, da die einzelnen \ac{FTF} nicht wie sonst üblich schienengebunden sind, sondern sich frei durch die Anlage bewegen können. Da es sich zudem mehrere Fahrzeuge gleichzeitig in der Anlage befinden können, muss durch geeignete Sensorik eine bevorstehende Kollision erkannt und verhindert werden.
\subsection{Dynamische Wegfindung}
	\label{Aufgabenstellung_Pathfinding}
	Für die Wegfindung ist es wichtig, dass das Werkstück die Ablaufreihenfolge der benötigten Arbeitsschritte kennt. Diese sollen in einer Art Rezept festgehalten werden und als Grundlage für die Berechnung der Teilrouten zu den Bearbeitungsstationen dienen. Für die Bestimmung der Routen wird zudem eine geeignete Beschreibung der Anlagentopologie benötigt. Innerhalb der Anlage können mehrere Bearbeitungsstationen die gleiche Funktionalität liefern. In diesem Fall soll die Wegfindung eine geeignete Station auswählen können Diese sollte universell gehalten sein, damit sie auch von dem \ac{FTS} genutzt werden kann. Die Wegfindung soll so realisiert sein, dass sie auch auf einer der leistungsschwächeren Steuerungen\footnote{\ac{SPS}} der Siemens S7-1200er Reihe lauffähig ist, falls dies nicht realisierbar ist, sollen so wenig Komponenten wie möglich auf ein PC-System ausgelagert werden.
	\\ Bei der Berechnung von Routen sollen zudem folgende dynamische Ereignisse berücksichtigt werden:
	\begin{itemize}
		\item Persistente Blockaden und länger andauernde  nicht permanente Änderungen der Standard-Anlagentopologie.
		\item Dynamische Blockaden von Teilstrecken durch andere Fahrzeuge zur Kollisionsvermeidung.
	\end{itemize}
	
	

\section{Erwartete Ziele für die Wegfindung}
	Da bei der Planung des Arbeitsauftrags noch nicht absehbar war, wie zeitaufwendig die Realisierung der Aufgabe sein würde, wurde die Funktionalität der Wegfindungskomponente der Anlage in mehrere Stufen unterteilt:\\
	
	\begin{tabular}{p{2.5cm} p{10cm}}
		\textbf{Grundstufe:} & Es können detaillierte Teilrouten generiert werden auf der Basis von Anfangs- und Endpunkten.\\[0.25cm]
		\textbf{Stufe 1:} & Es können komplette Routen bestehend aus Teilrouten anhand eines einzigen Fertigungsplans generiert werden, es existiert pro Funktionalität nur eine Bearbeitungsstation. Mehrere \ac{FTF} erhalten die gleiche Route und reihen sich hintereinander ein.\\[0.25cm]
		\textbf{Stufe 2:} & Es wird werden komplette Routen anhand eines einzigen Fertigungsplans generiert, pro Funktionalität existieren mehrere Bearbeitungsstationen. Die \ac{FTF} können unterschiedliche Wege nehmen und sich gegenseitig überholen.\\[0.25cm]
		\textbf{Stufe 3:} & Analog zu Stufe 2, es können jedoch im laufenden Betrieb Störungen in der Form von persistenten Blockaden auftreten, die bei der Wegfindung berücksichtigt werden.\\[0.25cm]
		\textbf{Stufe 4:} & Analog zu Stufe 3, verschiedene Fahrzeuge können unterschiedliche Fertigungspläne und somit andere Routen verfolgen.\\[0.25cm]
	\end{tabular}
	
	Als zusätzliches Ziel wurde die Visualisierung der geplanten und bereits zurückgelegten Fahrstrecken der verschieden \ac{FTF} definiert, zur besseren Darstellung der einzelnen Aspekte der definierten Stufen. Die Anlage wird zudem in Kooperation mit der Siemens Entwicklungsabteilung \ac{CT} entwickelt wird, welche parallel die gleiche Anlage als Simulation testet und die ermittelten Ergebnisse mit dem Resultat der physikalischen Anlage vergleicht.
	