\chapter{Definition der Anforderungen}

\section{Allgemeine Aufgabenstellung}
Diese Arbeit beschäftigt sich mit dem Entwurf und der Realisierung eines \ac{CPPS} im Modellmaßstab. Die resultierende Anlage soll unter anderem dazu dienen verschiedene Aspekte von Industrie 4.0 vorzuführen und zu veranschaulichen. Kernpunkte, die dargestellt werden sollen, sind vor allem die Dezentralisierung und Skalierbarkeit der Anlage. Den Rahmen für die Bearbeitung dieser Aufgabe bildet die Fachberatung für Automatisierungstechnik der Siemens AG in München. Um die erwähnten Konzepte demonstrieren zu können, soll die Anlage gemäß ihrer realen Vorbilder bestehen aus Bearbeitungsstationen, an welchen der Bearbeitungsprozess simuliert werden kann, und Werkstückträgern, welche Werkstücke durch die Modellanlage zu den Maschinenplätzen transportieren können. Im vorliegenden Fall stellen die Werkstückträger gleichzeitig das Werkstück dar, dass die Produktionsanlage durchfährt.

\subsection{Aufteilung der Themenbereiche}

\subsubsection{Fahrerloses Transportsystem}
\subsubsection{Dynamische Wegfindung}