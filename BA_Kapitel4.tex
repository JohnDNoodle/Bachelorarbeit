\chapter{Technische Implementierung der Algorithmen}

	\section{Kurze Einführung in die Programmiersprache und Programmierumgebung}
		Für die Implementierung der Wegfindung wurde im Kapitel \ref{Aufgabenstellung_Pathfinding} definiert, dass diese auf einer Siemens \ac{SPS} (englisch: \ac{PLC})  der S7-1200er Reihe lauffähig ist. Bei der 1200er Reihe handelt es sich um  Steuerungen des niedrigen Leistungssegments. Für die Projektierung und Programmierung von Anlagen mit Steuerungen dieser Art wird deine proprietäre Entwicklungsumgebung namens \ac{TIA-Portal} von Siemens zur Verfügung gestellt.
		\subsection{Siemens TIA-Portal}
			Das Siemens \ac{TIA-Portal} vereinigt viele Aspekte der Projektierung von Anlagen in einer einheitlichen Oberfläche. Innerhalb des \ac{TIA-Portal}s können beispielsweise Projekte bestehend aus mehreren Antrieben und \ac{SPS}en gemeinsam geplant und erstellt werden. Die aktuelle Version des \ac{TIA-Portal}s ist in der Version 13 verfügbar und bietet vor allem eine anwenderfreundliche Oberfläche für komplexe Automatisierungsaufgaben. Die Kernkomponente für die Erstellung von Programmen für \aclp{SPS} ist die Programmierumgebung \acs{STEP7}\footnote{\ac{STEP7}}. Abbildung \textbf{INSERT GRAPHIC HERE} zeigt die Projektansicht des \ac{TIA-Portal}s 
		
		\subsection{Programmierumgebung STEP7}
			Die Grundelemente eines \ac{STEP7}-Projekts sind die projektierten Steuerungen. Diese sind wiederum unterteilt in Teilelemente wie die Hardware-Konfiguration der Steuerung, das Anwenderprogramm, die verwendeten Variablen, benötigte Datentyp-Definitionen und Komponenten zur Überwachung und Modifizierung der Steuerungsdaten im laufenden Betrieb \textbf{Insert grafic here}. Für die Implementierung der Wegfindungsalgorithmen sind vor allem das Steuerungsprogramm und die darin verwendeten Datentypen von Bedeutung. Ein Anwenderprogramm besteht aus bis zu vier Arten von Programmbausteinen:
			
			\begin{tabular}{{p{4.5cm} p{10cm}}}
				
				\textbf{\ac{OB}:} & Diese Bausteine bilden die Schnittstelle zwischen dem Betriebssystem und dem Anwenderprogramm. Sie haben jeweils vordefinierte Funktionalitäten und bilden somit das Grundgerüst des Anwenderprogramms. Die in dieser Implementierung verwendeten \ac{OB}s sind zum einen der Systemstart-Baustein und der Baustein zur zyklischen Abarbeitung von Teilschichten des Programms.\\[0.5cm]
				\textbf{\ac{DB}:} & Datenbausteine dienen zur Speicherung von variablen Daten, die im gesamten Anwenderprogramm benötigt werden. Sie werden unter anderem zur Sicherung der Topologiedaten der Anlage, sowie als Schnittstellen zwischen verschiedenen Programmschichten verwendet.\\[0.5cm]
				\textbf{\ac{FC}:} & Funktionen sind Bausteine zur elementaren Kapselung von Funktionalitäten. Sie werden im Anwenderprogramm definiert als Unterprogramme, die keinen eigenen Speicher zur Sicherung von Variablenwerten zwischen zwei aufeinanderfolgenden Programmaufrufen benötigen.\\[0.5cm]
				\textbf{\ac{FB}:} & Funktionsbausteine realisieren wie \ac{FC}s Unterprogramme, stellen aber zusätzlichen Speicherbereich für die permanente Sicherung von Daten internen Variablen zur Verfügung. Bei der Verwendung eines \ac{FB}s wird bei dessen Initialisierung ein entsprechender Instanz-\ac{DB} generiert, in dem Daten für die Verwendung in späteren Programmaufrufen gespeichert werden können.\\[0.5cm]
				
			\end{tabular}
			
			\ac{FC}s und \ac{FB}s entsprechen den Funktionsdefinitionen in anderen Programmiersprachen. Es können die Schnittstellen der Bausteine sowie deren Schnittstellentypen definiert werden. IN-Variablen werden beispielsweise nur lesend verwendet, OUT-Variablen werden nur schreibend verwendet und INOUT-Variablen werden sowohl schreibend als auch lesend verwendet. Innerhalb eines Bausteins können sowohl temporäre als auch statische\footnote{persistent über Funktionsaufrufe hinaus} Variablen zur Zwischenspeicherung von Variablenwerten während der Programmabarbeitung genutzt werden. Da statische Variablen einen Instanz-\ac{DB} benötigen, sind sie nur in \ac{FB}s verwendbar.
			Die Bausteine können in einer von vier Programmiersprachen geschrieben werden. \ac{FUP} und \ac{KOP} sind Sprachen zur graphischen Programmierung. \ac{AWL} ist eine Assembler-ähnliche Sprache für generelle Programmieraufgaben, die unter anderem die byteweise Manipulation von Daten vereinfacht. \ac{SCL} ist eine Pascal-ähnliche Programmiersprache, die durch ihre einfachen Implementierungsmöglichkeiten von Schleifen geeignet ist für die Programmierung komplexer Aufgabenstellungen \textbf{insert comparison graphic here}. Bei der Erstellung des Anwenderprogramms für die Wegfindung wurden die Verwendeten \ac{OB}s in \ac{FUP} erstellt und alle anderen Bausteine in SCL.
			
		%\subsection{SCL}
	
		\subsection{Arbeitsweise einer SPS}
			
			Eine \acl{SPS} arbeitet nach dem Prinzip eines Echtzeitsystems. Das projektierte Anwenderprogramm wird in einer Endlosschleife zyklisch abgearbeitet. Zu Beginn eines Bearbeitungszyklus wird ein Prozessabbild aller Eingangsbaugruppen  der Steuerung generiert, das für den kompletten Zyklus als Basis für die Werte der Eingänge benutzt wird. Während des Zyklus werden die berechneten Werte für die Ausgänge in ein weiteres Prozessabbild geschrieben, welches erst nach Ende des aktuellen Bearbeitungszyklus an die Ausgangsbaugruppen übertragen wird. Somit müssen Mehrfachzuweisungen innerhalb eines Zyklus vermieden werden, da nur die letzte Zuweisung an die Ausgänge weitergegeben wird \textbf{insert PAE PAA graphic here}. Durch \ac{OB}s können zusätzliche Funktionen außerhalb der zyklischen Bearbeitung realisiert werden. Beispielsweise können im Startup-\ac{OB} einmalig Anweisungen beim Hochfahren der CPU ausgeführt werden.
	
	\section{Beschreibung der Anlagentopologie}

	\section{Implementierung des Dijkstra-Algorithmus}

	\section{Implementierung des A*-Algorithmus}
	
	\section{Einhaltung der Echtzeitbedingung}
	
		\cite{BorisCherkassky1993}
	
		\subsection{Ausführung bei Systemstart}
		
		\subsection{Zyklische Ausführung}