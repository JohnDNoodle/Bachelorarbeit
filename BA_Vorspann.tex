\pagenumbering{Roman}
\begin{titlepage}
	\begin{figure}[h]
		 
		\begin{minipage}{0.45\linewidth} 
			\raggedright 
			\includegraphics[scale=0.15]{HMLogo.png} 
		\end{minipage} 
		\begin{minipage}{0.45\linewidth} 
			\raggedleft
			\vspace*{1 cm} 
			\includegraphics[scale=0.4]{sielogopetrolcmyk.jpg} 
		\end{minipage} 
	\end{figure}
	\begin{center} 
		\vspace{1.5 cm} {\LARGE Hochschule München} \\ 
		\vspace{1 cm} {\LARGE Fakultät für Elektrotechnik und Informationstechnik} \\ 
		\vspace{2 cm} {\Large Studiengang Elektrotechnik und Informationstechnik} \\ 
		\vspace*{2 cm}{\Huge \bf Industrie 4.0 - Pathfinding auf einer SPS\\} 
		\vspace{2 cm}
		{\bf Bachelorarbeit von Niels Garibaldi}\\
	\end{center}
	\vspace{4 cm}
	\begin{center}
		
		\begin{tabular}{ll}
			
			{\bf Bearbeitungsbeginn:} & 22.04.2016\\
			{\bf Abgabetermin:} & 15.09.2016\\
			{\bf lfd. Nr.:} & 1351\\
		\end{tabular}
	\end{center}

\end{titlepage} 

\begin{titlepage}
	\begin{figure}[h]
		
		\begin{minipage}{0.45\linewidth} 
			\raggedright 
			\includegraphics[scale=0.15]{HMLogo.png} 
		\end{minipage} 
		\begin{minipage}{0.45\linewidth} 
			\raggedleft
			\vspace*{1 cm} 
			\includegraphics[scale=0.4]{sielogopetrolcmyk.jpg} 
		\end{minipage} 
	\end{figure}
	\begin{center} 
		\vspace{1.5 cm} {\LARGE Hochschule München} \\ 
		\vspace{1 cm} {\LARGE Fakultät für Elektrotechnik und Informationstechnik} \\ 
		\vspace{2 cm} {\Large Studiengang Elektrotechnik und Informationstechnik} \\ 
		\vspace*{2 cm}{\Huge \bf Industrie 4.0 - Pathfinding auf einer SPS\\} 
		\vspace{1 cm}
		\hrule
		\vspace*{1 cm}{\Huge \bf Industry 4.0 - Pathfinding on a PLC\\} 
		\vspace{2 cm}
		{\bf Bachelorarbeit von Niels Garibaldi}\\
		\vspace{1 cm}
		{\bf betreut von  Prof. Dr. K. Ressel}\\
		
	\end{center}
	\vspace{1 cm}
	\begin{center}
		
		\begin{tabular}{ll}
			
			{\bf Bearbeitungsbeginn:} & 22.04.2016\\
			{\bf Abgabetermin:} & 15.09.2016\\
			{\bf lfd. Nr.:} & 1351\\
		\end{tabular}
	\end{center}
	
\end{titlepage}
\pagestyle{headings}
\clearpage
\phantomsection
\addcontentsline{toc}{section}{Erklärung des Bearbeiters}
\section*{Erklärung des Bearbeiters}



\begin{enumerate}
	\vspace{2 cm}
	\item
		Ich erkläre hiermit, dass ich die vorliegende Bachelorarbeit selbständig
		verfasst und noch nicht anderweitig zu Prüfungszwecken vorgelegt habe.\\
		
		Sämtliche benutzte Quellen und Hilfsmittel sind angegeben, wörtliche
		und sinngemäße Zitate sind als solche gekennzeichnet.\\
	
	\item
		Ich erkläre mein Einverständnis, dass die von mir erstellte Bachelorarbeit in die Bibliothek
		der Hochschule München eingestellt wird. Ich wurde darauf hingewiesen, dass die
		Hochschule in keiner Weise für die missbräuchliche Verwendung von Inhalten durch Dritte
		infolge der Lektüre der Arbeit haftet. Insbesondere ist mir bewusst, dass ich für die
		Anmeldung von Patenten, Warenzeichen oder Geschmacksmustern selbst verantwortlich
		bin und daraus resultierende Ansprüche selbst verfolgen muss. 
	
	\vspace{6 cm}
\end{enumerate}
\begin{flushright}
	\begin{tabular}{cl}
		München, den 14.09.2016, & \_\_\_\_\_\_\_\_\_\_\_\_\_\_\_\_\_\_\_\_\_\_\_\_\_\_\_\_\_\_\_\_\_\_\_\_\_\_\tabularnewline
		& Niels Garibaldi\tabularnewline
	\end{tabular}
	\par
\end{flushright}
\clearpage

%% Kurzfassung/Abstract
\phantomsection
\addcontentsline{toc}{section}{Zusammenfassung/Abstract}	
\begin{abstractDuo}
	Die Wichtigkeit von Industrie 4.0-Aspekten in modernen Produktionsprozessen erfordert auch zunehmend komplexere Algorithmen auf industriellen Steuerungen. Diese Arbeit zeigt, dass es auch auf den kleinsten Steuerungen möglich ist, komplexe Wegfindungsalgorithmen zu implementieren. Durch die geeignete Wahl des Algorithmus und der zugehörigen Heuristik können die Hindernisse des begrenzten Speicherplatzes und der Echtzeitanforderung einer speicherprogrammierbaren Steuerung überwunden werden. Aufbauend hierauf wurde eine Modellanlage mit fünf fahrerlosen Transportfahrzeugen gebaut, welche sich anhand der Wegfindung autonom durch die Anlage bewegen und so einen Produktionsprozess simulieren können. Die Wegfindung selbst ist dezentralisiert aufgebaut, was die Ausfallsicherheit erhöht. Die Anlage verfügt über eine drahtloses Kommunikationssystem, mithilfe welchem Daten über die verschiedenen Fahrzeuge ausgetauscht werden. Somit können die Fahrzeuge dynamisch aufeinander reagieren und ausweichen, um Kollisionen zu verhindern. Die finale Anlage dient zu Vorführzwecken und soll  gemäß des Konzepts eines Cyberphysischen Produktionssystems an eine überlagerte Simulation angebunden werden.
\end{abstractDuo}
\selectlanguage{english}
\begin{abstractDuo}
	The importance of aspects of Industry 4.0 in modern manufacturing processes shows the need for increasingly complex algorithms even on industrial controllers. This paper shows, that it is possible to implement complex pathfinding algorithms even on the smallest controllers. Through the selection of a suitable algorithm and corresponding heuristics, one can overcome the obstacles of restricted memory and real-time constraints that come with a programmable logic controller. Based on this, a model factory was constructed, which consists of five automated guided vehicles, that autonomously path their way through the factory and simulate the manufacturing process. The pathfinding system is decentralized to improve its reliability. Through a wireless communication system, the different vehicles exchange data to dynamically react to each other and avoid collisions. The finalized factory shall be connected to an overlying simulation, in order to show the Industry 4.0-concept of a cyber-physical production system.
\end{abstractDuo}
\selectlanguage{ngerman}
\clearpage
%% Inhaltsverzeichnis
\pagestyle{plain}
\clearpage
\phantomsection
\addcontentsline{toc}{section}{\contentsname}

	\begin{spacing}{1.1}
	
		\tableofcontents
	\end{spacing}


\clearpage
\phantomsection
\addcontentsline{toc}{section}{Abkürzungsverzeichnis}	
\section*{Abkürzungsverzeichnis}
\begin{acronym}[Bash]
	
	\acro{AWL}{Anweisungsliste}
	\acro{CPS}{Cyber-physisches System}
	\acro{CPPS}{Cyber-physisches Produktionssystem}
	\acrodefplural{CPPS}{Cyber-physische Produktionssysteme}
	\acro{CT}{Corporate Technologies}
	\acro{D*}{Focussed Dynamic A*}
	\acro{DB}{Datenbaustein}
	\acro{FB}{Funktionsbaustein}
	\acro{FC}{Funktion}
	\acro{FTF}{Fahrerloses Transportfahrzeug}
	\acro{FTS}{Fahrerloses Transportsystem}
	\acro{FUP}{Funktionsplan}
	\acro{IoT}{Internet of Things}
	\acro{KOP}{Kontaktplan}
	\acro{MA*}{Memory Bounded A*}
	\acro{MES}{Manufacturing Execution System}
	\acro{OB}{Organisationsbaustein}
	\acro{PLC}{Programmable Logic Controller}
	\acro{RFID}{Radio Frequency Identification}
	\acro{RTA*}{Real-Time A*}
	\acro{SCL}{Structured Control Language}
	\acro{SPS}{Speicherprogrammierbare Steuerung}
	\acrodefplural{SPS}{Speicherprogrammierbare Steuerungen}
	\acro{STEP7}{STeuerungen Einfach Programmieren Version 7}
	\acro{TCP}{Transport Control Protocol}
	\acro{TIA-Portal}{Totally Integrated Automation Portal}
	\acro{UDP}{User Datagramm Protocol}
	\acro{WLAN}{Wireless Local Area Network}

\end{acronym}

%%Nomenclature definitions


%	\nomenclature{CPS}{Cyber-physisches System}
%	\nomenclature{CPPS}{Cyber-physisches Produktionssystem}
%	\nomenclature{IoT}{Internet of Things}
%	\nomenclature{PLC}{Programmable Logic Controller}
%	\nomenclature{SPS}{Speicherprogrammierbare Steuerung}




\clearpage
\pagestyle{headings}
\pagenumbering{arabic}
